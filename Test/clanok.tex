% Metódy inžinierskej práce

\documentclass[10pt,twoside,slovak,a4paper]{article}

\usepackage[slovak]{babel}
%\usepackage[T1]{fontenc}
\usepackage[IL2]{fontenc} % lepšia sadzba písmena Ľ než v T1
\usepackage[utf8]{inputenc}
\usepackage{graphicx}
\usepackage{url} % príkaz \url na formátovanie URL
\usepackage{hyperref} % odkazy v texte budú aktívne (pri niektorých triedach dokumentov spôsobuje posun textu)

\usepackage{cite}
%\usepackage{times}

\pagestyle{headings}

\title
{
	Rozvoj e-learningu a stratégie využívané vonline vyučovaní
	\thanks
	{
		Semestrálny projekt v predmete Metódy inžinierskej práce, ak. rok 2020/21, vedenie: Michal Hatala
	}
} % meno a priezvisko vyučujúceho na cvičeniach

\author{Michal Magula\\[2pt]
	{\small Slovenská technická univerzita v Bratislave}\\
	{\small Fakulta informatiky a informačných technológií}\\
	{\small \texttt{...@stuba.sk}}
	}

\date{\small 29. október 2020}



\begin{document}

\maketitle

\section{Ako sa vyvíjalo e-vzdelávanie} \label{Evolution}

	Mnoho ľudí si mýli pojem e-vzdelávanie s distančným vzdelávaním a myslia si, že sú tieto slová synonymami.
	V skutočnosti sú to dva rozdielne pojmy. Vývoj e-vzdelávania nastal v 90. tych rokoch spolu s rozvojom internetu.
	Samozrejme nemôžeme poprieť, že e-learning nemá svoje korene v dištančnom vzdelávaní.

	Prvé náznaky dištančného vzdelávania datujeme v roku 1828, keď Profesor C. Phillips inzeroval do novín
	Boston Gazette ponuku na učebné materiály a tutoriály odosielané poštou. V roku 1843 bola 
	založená Phonographic Correspondence Society, ktorá by mohla byť považovaná za prvú inštitúciu 
	venujúcu sa dištančnému vzdelávaniu. Žiaci, ktorí sa zúčastnovali ich kurzu dostávali poštou
	nové a opravené zadania.

	V 20. tych rokoch 20. teho storočia s príchodom masovo komunikačných prostriedkov sa začala formovať aj idea
	vzdelávania žiakov cez tieto prostriedky. Tento nápad bol uskútočnený až v 60. tych tokoch minulého storočia 
	vznikom Open University in UK (Otvorenej univerzity v Spojenom Kráľovstve).

	Koncep e-vzdelávania sa vyvíjal ruka v ruke spolu s rozvojom počítačov, ale 
	najväčšií mýlnik pre e-vzdelávanie bol vznik webu. Od tohto momentu sa 
	e-vzdelávanie začalo rozvíjať extrémne rýchlo.



\section{Generácie e-vzdelávania} \label{Generations}

Dalšia sekcia



\section{Pedagogické postupy v e-vzdelávanií} \label{pedagogicalApproaches}

Tu budem písať o nejakých využívaných pedagogických postupoch.

\section{Ekosystémy vzdelávania} \label{ecosystems}



\section{Záver} \label{zaver} 

%\acknowledgement{Ak niekomu chcete poďakovať\ldots}


% týmto sa generuje zoznam literatúry z obsahu súboru literatura.bib podľa toho, na čo sa v článku odkazujete
\bibliography{literatura}{}
\bibliographystyle{plain} % prípadne alpha, abbrv alebo hociktorý iný
\end{document}
