\documentclass[12pt,twoside,slovak,a4paper]{article}

\usepackage[slovak]{babel}
\usepackage[IL2]{fontenc}
\usepackage[utf8]{inputenc}



\begin{document}
    \begin{abstract}
        Toto je abstrakt
    \end{abstract}
    \section{Ako sa e-vzdelávanie vyvíjalo}
    Mnoho ľudí si mýli pojem e-vzdelávanie s distančným vzdelávaním a myslia si, že sú tieto slová synonymami.
    V skutočnosti sú to dva rozdielne pojmy. Vývoj e-vzdelávania nastal v 90. tych rokoch spolu s rozvojom internetu.
    Samozrejme nemôžeme poprieť, že e-learning nemá svoje korene v dištančnom vzdelávaní.

    Prvé náznaky dištančného vzdelávania datujeme v roku 1828, keď Profesor C. Phillips inzeroval do novín
    Boston Gazette ponuku na učebné materiály a tutoriály odosielané poštou. V roku 1843 bola 
    založená Phonographic Correspondence Society, ktorá by mohla byť považovaná za prvú inštitúciu 
    venujúcu sa dištančnému vzdelávaniu. Žiaci, ktorí sa zúčastnovali ich kurzu dostávali poštou
    nové a opravené zadania.

    V 20. tych rokoch 20. teho storočia s príchodom masovo komunikačných prostriedkov sa začala formovať aj idea
    vzdelávania žiakov cez tieto prostriedky. Tento nápad bol uskútočnený až v 60. tych tokoch minulého storočia 
    vznikom Open University in UK (Otvorenej univerzity v Spojenom Kráľovstve).
\end{document}