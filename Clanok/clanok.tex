\documentclass[10pt,twoside,slovak,a4paper]{article}

\usepackage[slovak]{babel}
\usepackage[IL2]{fontenc}
\usepackage[utf8]{inputenc}

\pagestyle{headings}

\title{Názov\thanks{Semestrálny projekt v predmete Metódy inžinierskej práce, ak. rok 2015/16, vedenie: Meno Priezvisko}} % meno a priezvisko vyučujúceho na cvičeniach

\author{Michal Magula\\[2pt]
	{\small Slovenská technická univerzita v Bratislave}\\
	{\small Fakulta informatiky a informačných technológií}\\
	{\small \texttt{...@stuba.sk}}
	}

\date{\small 29. október 2020}

\begin{document}

    \begin{abstract}
        Za posledné roky sa e-learning veľmi spopularizoval. Dôvodom je najmä zlacňovanie osobnej elektroniky
        a výrazné zlepšovanie internetového pokrytia. Hoci týmto systémom je možné sa vzdelávať kedykoľvek
        a kdekoľvek, je stále potrebné vyskúšať mnoho postupov výučby a až časom zistíme, čo je najvhodnejšie.
        Preto účelom tohto článku je oboznámiť čitateľa ohľadom problematiky e-learningu, z čoho e-learning
        vznikol a ako sa vyvinul do podoby v akej ho poznáme dnes. Ďalším cieľom je poskytnúť informácie práve
        ohľadom trendov a stratégií využívaných v online výučbe.
    \end{abstract}

    \section{Ako sa e-vzdelávanie vyvíjalo}
        Mnoho ľudí si mýli pojem e-vzdelávanie s distančným vzdelávaním a myslia si, že sú tieto slová synonymami.
        V skutočnosti sú to dva rozdielne pojmy. Vývoj e-vzdelávania nastal v 90. tych rokoch spolu s rozvojom internetu.
        Samozrejme nemôžeme poprieť, že e-learning nemá svoje korene v dištančnom vzdelávaní.

        Prvé náznaky dištančného vzdelávania datujeme v roku 1828, keď Profesor C. Phillips inzeroval do novín
        Boston Gazette ponuku na učebné materiály a tutoriály odosielané poštou. V roku 1843 bola 
        založená Phonographic Correspondence Society, ktorá by mohla byť považovaná za prvú inštitúciu 
        venujúcu sa dištančnému vzdelávaniu. Žiaci, ktorí sa zúčastnovali ich kurzu dostávali poštou
        nové a opravené zadania.

        V 20. tych rokoch 20. teho storočia s príchodom masovo komunikačných prostriedkov sa začala formovať aj idea
        vzdelávania žiakov cez tieto prostriedky. Tento nápad bol uskútočnený až v 60. tych tokoch minulého storočia 
        vznikom Open University in UK (Otvorenej univerzity v Spojenom Kráľovstve).
\end{document}